\documentclass[a4paper,11pt]{article}
\usepackage{zh_CN-Adobefonts_external} % Simplified Chinese Support using external fonts (./fonts/zh_CN-Adobe/)
\usepackage{fancyhdr}  % 页眉页脚
\usepackage{minted}    % 代码高亮
\usepackage[colorlinks]{hyperref}  % 目录可跳转
\setlength{\headheight}{15pt}

% 定义页眉页脚
\pagestyle{fancy}
\fancyhf{}
\fancyhead[C]{Algorithm Template}
\lfoot{}
\cfoot{\thepage}
\rfoot{}

\author{Kevco}   
\title{Algorithm Template}

\begin{document} 
\maketitle % 封面
\newpage % 换页
\tableofcontents % 目录
\newpage

\section{DP} % 一级标题


\subsection{背包DP} % 二级标题

\subsubsection{01背包} % 三级标题
\inputminted[breaklines]{c++}{dp/bag/01.cpp}
\newpage % 换页
\subsubsection{完全背包} % 三级标题
\inputminted[breaklines]{c++}{dp/bag/wanQuan.cpp}
\newpage % 换页
\subsubsection{多重背包} % 三级标题
\inputminted[breaklines]{c++}{dp/bag/duoChong1.cpp}
\inputminted[breaklines]{c++}{dp/bag/duoChong2.cpp}
\inputminted[breaklines]{c++}{dp/bag/duoChong3.cpp}
\newpage % 换页
\subsubsection{分组背包} % 三级标题
\inputminted[breaklines]{c++}{dp/bag/fenZu.cpp}
\newpage % 换页
\subsubsection{树上背包} % 三级标题
\inputminted[breaklines]{c++}{dp/bag/treeBag1.cpp}
\newpage % 换页
\inputminted[breaklines]{c++}{dp/bag/treeBag3.cpp}
\newpage % 换页
\inputminted[breaklines]{c++}{dp/bag/treeBag2.cpp}
\newpage % 换页
\subsubsection{k优背包} % 三级标题
\inputminted[breaklines]{c++}{dp/bag/k-great.cpp}
\newpage % 换页
\subsubsection{前后缀背包} % 三级标题
\inputminted[breaklines]{c++}{dp/bag/presuf.cpp}
\newpage % 换页
\subsubsection{回退背包} % 三级标题
\inputminted[breaklines]{c++}{dp/bag/huitui.cpp}
\newpage % 换页
\subsubsection{方案计数} % 三级标题
\inputminted[breaklines]{c++}{dp/bag/count1.cpp}
\inputminted[breaklines]{c++}{dp/bag/count2.cpp}
\newpage % 换页
\subsubsection{输出方案} % 三级标题
\inputminted[breaklines]{c++}{dp/bag/getMethod.cpp}
\newpage % 换页
\subsubsection{DP逆推} % 三级标题
\inputminted[breaklines]{c++}{dp/bag/inv.cpp}
\newpage % 换页
\subsubsection{其他} % 三级标题
\inputminted[breaklines]{c++}{dp/bag/special.cpp}

\newpage % 换页
\subsection{线性DP} % 二级标题

\subsubsection{LIS} % 三级标题
\inputminted[breaklines]{c++}{dp/线性dp/LIS1.cpp}
\inputminted[breaklines]{c++}{dp/线性dp/LIS2.cpp}
\newpage % 换页
\subsubsection{LCS} % 三级标题
\inputminted[breaklines]{c++}{dp/线性dp/LCS.cpp}
\newpage % 换页
\subsubsection{LCIS} % 三级标题
\inputminted[breaklines]{c++}{dp/线性dp/LCIS.cpp}
\newpage % 换页
\subsubsection{回文字符串} % 三级标题
\inputminted[breaklines]{c++}{dp/线性dp/string.cpp}
\newpage % 换页
\subsubsection{方格取数} % 三级标题
\inputminted[breaklines]{c++}{dp/线性dp/matrix.cpp}

\newpage % 换页
\subsection{树形DP} % 二级标题

\subsubsection{树的中心} % 三级标题
\inputminted[breaklines]{c++}{dp/树形/center.cpp}
\newpage % 换页
\subsubsection{最大深度和} % 三级标题
\inputminted[breaklines]{c++}{dp/树形/maxdepth.cpp}

\newpage % 换页
\subsection{状压DP} % 二级标题

\subsubsection{例l} % 三级标题
\inputminted[breaklines]{c++}{dp/状压/1.cpp}

\newpage % 换页
\subsection{数位DP} % 二级标题

\subsubsection{例l} % 三级标题
\inputminted[breaklines]{c++}{dp/数位/windy.cpp}

\newpage % 换页
\subsection{概率DP} % 二级标题

\subsubsection{例l} % 三级标题
\inputminted[breaklines]{c++}{dp/概率dp/1.cpp}

\newpage % 换页
\subsection{子集和DP} % 二级标题

\subsubsection{枚举子集超集} % 三级标题
\inputminted[breaklines]{c++}{dp/子集和dp/1.cpp}
\newpage % 换页
\subsubsection{子集计数} % 三级标题
\inputminted[breaklines]{c++}{dp/子集和dp/2.cpp}
\newpage % 换页
\subsubsection{其他例子} % 三级标题
\inputminted[breaklines]{c++}{dp/子集和dp/3.cpp}
\newpage % 换页

%\newpage
%\section{Others}

\end{document}
